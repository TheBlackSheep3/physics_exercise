\section{Thermodynamik idealer Gase}
\begin{auf}
    360
\end{auf}
Luft vom Volumen $V_1=50cm^3$ und der Temperatur $T_1=300K$ soll bei konstantem Druck	von $p=1bar$ auf $T_2=1000K$ erwärmt werden.
\begin{enumerate}
    \item[a] Skizzieren Sie die Zustandsänderung in einem $p,V$-Diagramm.
    \item[b] Berechnen Sie das Endvolumen $V_2$.
    \item[c] Berechnen Sie die verrichtete Ausdehnungsarbeit $W_{12}$.
    \item[d] Berechnen Sie die Änderung der inneren Energie des Gases $\Delta U_{12}$, Adiabatenexponent $\kappa=\frac{c_p}{c_V}=1.4$
\end{enumerate}