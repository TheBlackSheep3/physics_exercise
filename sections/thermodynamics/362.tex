\begin{auf}
    362
\end{auf}
Luft (Molmasse $M\approx29\cdot10^{-3}\frac{kg}{mol}$) wird von einem Kompressor bei Atmosphärendruck $p_1=1bar$ und bei der Temperatur $T=294K$ angesaugt und auf den Druck $p_2=35bar$ verdichtet.
\begin{enumerate}
    \item[a] Wie groß ist die am Gas verrichtete Volumenarbeit je Kilogramm komprimierter Luft, wenn die Kompression isotherm erfolgen soll?
    \item[b] Welche Wärme $Q$ je Kilogramm Luft wird dabei an das Kühlwasser abgegeben?
\end{enumerate}