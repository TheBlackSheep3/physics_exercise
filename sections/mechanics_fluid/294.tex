\begin{auf}
    294
\end{auf}
Ein waagrechtes Rohr, das sich auf ein Drittel seines Durchmessers verjüngt, wird von Wasser ($\varrho = 10\frac{kg}{m^3}$) durchströmt. In den beiden unterschiedlich dicken Teilen des Rohres besteht eine (statische) Druckdifferenz von $6.4kPa$.
\begin{enumerate}
    \item[a] Wie groß sind die Strömungsgeschwindigkeiten in beiden Querschnitten?
    \item[b] Wie groß ist der Massenstrom $I_m$ für $D_1=12cm$ (Durchmesser des weiten Rohrteils)?
\end{enumerate}