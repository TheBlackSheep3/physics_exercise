\section{Formelsammlung}
\subsection*{Gleichmäßig beschleunigte Bewegung}
Für die Geschwindigkeit $v$ in Abhängigkeit der Zeit $t$ mit einer konstanten Beschleunigung $a$ gilt:
\begin{align*}
    v=a\cdot t+v_0
\end{align*}
Für den Weg $s$ in Abhängigkeit der Zeit gilt:
\begin{align*}
    s=\frac{a}{2}\cdot t^2+v_0\cdot t+s_0
\end{align*}
Aus diesen beiden Gleichungen ergibt sich:
\begin{align*}
    v^2-v_0^2=2\cdot a\cdot s
\end{align*}
\subsection*{Elastischer Stoß}
Für zwei Körper mit den Massen $m_1$, $m_2$ und den Anfangsgeschwindigkeiten $v_1$, $v_2$ gilt:
\begin{align*}
    m_1\cdot v_1+m_2\cdot v_2=m_1\cdot v'_1+m_2\cdot v'_2
\end{align*}
\subsection*{Unelastischer Stoß}
Für zwei Körper mit den Massen $m_1$, $m_2$ und den Anfangsgeschwindigkeiten $v_1$, $v_2$ gilt:
\begin{align*}
    m_1\cdot v_1+m_2\cdot v_2=(m_1+m_2)\cdot u
\end{align*}
\subsection*{Drehmoment}
Greift an einem Körper eine Kraft $F$ an mit dem Abstand $r$ zum Schwerpunkt an entsteht folgendes Drehmoment $M$:
\begin{align*}
    M=r\cdot F
\end{align*}
Für einen Körper mit dem Drehmoment $M$, dem Trägheitsmoment $J$ und der Winkelbeschleunigung $\alpha$ gilt folgende Beziehung, welche auch als Grundgesetz der Dynamik bezeichnet wird.
\begin{align*}
    M=J\cdot \dot{\omega}=J\cdot \alpha
\end{align*}
\subsection*{Gegenüberstellung Translation Rotation}
\begin{center}
    \begin{tabular}{l l | l l}
        Translation & & Rotation & \\
        \hline
        Weg & $s$ & Drehwinkel & $\varphi$\\
        Geschwindigkeit & $v=\dot{s}$ & Winkelgeschwindigkeit & $\omega = \dot \varphi$\\
        Beschleunigung & $a=\dot{v}=\ddot{s}$ & Winkelbeschleunigung & $\alpha=\dot{\omega}=\ddot{\varphi}$\\
        Masse & $m$ & Trägheitsmoment & $J$\\
        Impuls & $p=m\cdot v$ & Drehimpuls & $L=J\cdot \omega$\\
        Kraft & $F=m\cdot a$ & Drehmoment & $M=J\cdot \alpha$\\
        Arbeit\footnotemark & $W=F\cdot s$ & Arbeit\footnotemark[1] & $W=M\cdot \varphi$\\
        Leistung\footnotemark[1] & $P=F\cdot v$ & Leistung\footnotemark[1] & $P=M\cdot\omega$\\
        kinetische Energie & $E_k=\frac{m}{2}v^2$ & Rotationsenergie & $E_k=\frac{J}{2}\omega^2$
    \end{tabular}
\end{center}
\footnotetext{Beziehungen gelten nur für einen speziellen Fall}